% !TEX root = ../intro-stellar-physics.tex

The depletion of hydrogen in the core heralds the end of the star's placid main-sequence life. Fusion of helium requires temperatures $\gtrsim\val{10^{8}}{\K}$, substantially higher than those required for fusion of hydrogen. As a consequence, when the hydrogen is used up, the core begins to contract, just as before the star joined the main sequence; the main difference is that while the core is contracting, hydrogen burning is still occurring in a shell surrounding the core. This causes drastic changes to the star's structure, surface temperature, and luminosity.

Once the core becomes sufficiently hot, helium fuses into carbon, and the core again reaches a state of thermal and mechanical equilibrium. When the helium is depleted the core must again contract. As with pre-main sequence stars, the critical question is whether the core becomes degenerate before a particular reaction can ignite. For stars with main-sequence masses $\lesssim \valrng[--]{8}{10}{\Msun}$, the core becomes degenerate before the onset of $\carbon$ fusion, which requires temperatures $\approx\val{\sci{8}{8}}{\K}$.  Indeed, for stars around a solar mass, the fusion of \helium\ occurs under moderately degenerate conditions.\sidenote{Stars with masses $\lesssim\val{0.5}{\Msun}$ will become degenerate before reaching temperatures sufficient for helium to fuse; the main-sequence lifetime of such stars is much greater than the age of the universe, however, making this of academic interest only.}
As a result, the cores of low-mass stars end up composed of carbon and oxygen and supported by degenerate electrons; such objects are known as \newterm{white dwarfs}.

For stars with higher masses, the core will fuse elements up through \iron. At this point the matter reaches its maximum binding energy\sidenote{cf.\ exercise~\ref{ex.nuclear-landscape}}. A degenerate core forms and grows in mass due to reactions in shells surrounding the core. There is a maximum mass, known as the \newterm{Chandrasekhar mass}, that can be supported by degeneracy pressure. When the core exceeds this mass, it violently implodes. This implosion stops when matter reaches nuclear density and the repulsive part of the strong nuclear force provides pressure support. In this implosion, most of the electrons and protons combine, $e^{-} + \pt \to \nt+\nu_{e}$. The core is then composed mostly of neutrons and is known as a \newterm{neutron star}.
The flood of neutrinos inject energy into the outer layers of the star; in many cases this is sufficient to eject the outer layers of the star and produce a \newterm{suprenova}. If the envelope is not ejected, matter will accumulate onto the neutron star. The maximum mass that can be supported by the nuclear force is $\val{2}{\Msun}<M_{\max}<\val{3}{\Msun}$; when this maximum mass is exceeded, the neutron star collapses into a black hole.

Having given a brief summary of post-main sequence evolution, we shall now explore the various evolutionary tracks in slightly more detail.

\section{Low-mass stars}

\subsection{Ascent of the red-giant branch}

With the depletion of hydrogen, the core contracts while there is \hydrogen\ fusion occurring in a shell surrounding the core. The shell hydrogen fusion produces helium, which adds to the core mass. As the core contracts its temperature rises. The rising temperature and pressure at the base of the hydrogen-burning shell causes the \hydrogen\ fusion reactions in the shell to go at an ever increasing rate. The resulting high luminosity drives the envelope, now fully convective, to large radii and hence to a low surface temperature. Observationally, the star becomes a red giant.  During this phase the high luminosity, combined with the low surface gravity of the distended envelope, drives a strong wind\sidenote{Calculations of the rate of mass loss are still crude, but there are observational constraints on the rate.}.

\subsection{Helium burning: the horizontal branch}
There are no stable isotopes with mass $A=5$ or $A=8$, which makes the fusion of \helium\ somewhat tricky.  The isotope \beryllium[8] is, however, long-lived ($\val{10^{-16}}{\second}$) compared to a nuclear timescale\sidenote{Roughly the time for a pion to cross a nucleus, $\sim \val{10^{-22}}{\second}$.}.
When the core temperature reaches $\approx\val{10^{8}}{\K}$, the reaction
\[ \helium + \helium \longleftrightarrow\beryllium[8] \]
builds up a small abundance of \beryllium[8]; the reaction
\[ \beryllium[8] + \helium \longleftrightarrow \carbon^{*} \]
then builds up a small abundance of \carbon\ in an excited state (denoted by the $^{*}$).  While most of $\carbon^{*}$ decays back into $\beryllium[8]+\helium$, a small fraction transitions to the ground state, $\carbon^{*}\to\carbon+\gamma$. As a result of these reactions, there is a net conversion $3\helium\to\carbon$---the \newterm{triple-alpha reaction}.
%This reaction is \emph{very temperature-sensitive}: $\partial\ln\epsilon_{3\alpha}/\partial\ln T \approx 40$ at $T = \val{10^{8}}{\K}$; this sensitivity, combined with the mildly degenerate conditions of the core, makes the ignition of \helium\ somewhat unstable for solar-mass stars.

Once core \helium\ has ignited, the star settles onto a ``helium main sequence;'' observationally this is the \newterm{horizontal branch}, so called because these stars form a noticeable clump on a Hertzsprung-Russell diagram. The luminosity on the horizontal branch is about \valrng[--]{30}{100}{\Lsun}. The higher luminosity, combined with the much lower energy release from the $3\helium\to\carbon$, makes the horizontal branch lifetime much shorter than that of the main-sequence (e.g., the horizontal branch lifetime is $\sim \val{10^{8}}{\yr}$ for a solar-mass star).

\begin{exercisebox}[Horizontal branch lifetime]
Following exercises \ref{ex.Q-hydrogen-helium} and \ref{ex.nuclear-burning-timescale}, find the heat released per kilogram from fusing 3 \helium\ nuclei ($B=\val{28.296}{\MeV}$) into \carbon\ ($B=\val{92.162}{\MeV}$). Take the core mass to be $\val{0.45}{\Msun}$ (the minimum core mass needed for the ignition of helium). For a luminosity of $\val{30}{\Lsun}$, find the lifetime for core helium burning.
\end{exercisebox}

As the mass of \carbon\ builds up in the core, the reaction $\carbon+\helium\to\oxygen$ begins to compete with the triple alpha reaction. As a result, the core becomes composed of a \carbon/\oxygen\ mixture.

\subsection{The asymptotic giant branch and emergence of a white dwarf}

With the depletion of \helium\ in the core, the core---now composed of \carbon\ and \oxygen---again contracts, while the growing luminosity from the H- and He-burning shells again drive the envelope to large radii. Observationally, this phase is the \newterm{asymptotic giant branch}: on an HR diagram, the stars move on a track that approaches the giant branch. The hydrogen-rich envelope is consumed at its base by the H- and He-burning shells and is expelled at the surface by an increasingly strong wind. After the envelope is gone, the hot core---observed as a white dwarf---slowly cools. For a solar-mass star, the expected final mass of the core, and hence of the white dwarf, is $\approx\val{0.6}{\Msun}$.

\section{Massive stars}

For stars with main-sequence masses $\gtrsim \valrng[--]{8}{10}{\Msun}$, the fusion of \carbon\ commences while the core is non-degenerate.  The temperature required is $\approx\val{\sci{8}{8}}{\K}$.  At this temperature, electron-positron pairs can form and annihilate ($e^{-}+e^{+}\longleftrightarrow\gamma\gamma$); occasionally instead of producing photons, the reaction
\[ e^{-}+e^{+} \longleftrightarrow \nu_{e} + \bar{\nu}_{e}\]
occurs instead and generates a neutrino-antineutrino pair. The mean free path for the neutrinos is larger than the radius of the star; as a result, the neutrinos stream out and take energy from the core. These neutrinos carry away most of the heat from the core.

Within the core, \carbon\ is consumed by
\[ \carbon+\carbon\to\left\{\begin{array}{c}\sodium+\pt \\\neon+\pt\end{array}\right.. \]
The \pt\ and \helium\ capture onto other nuclei that are present.  At slightly higher temperatures, the reaction $\neon+\gamma \to \oxygen+\helium$ occurs; the \helium\ capture onto other \oxygen, \neon, and \magnesium. The next significant burning stage is
\[\oxygen+\oxygen\to\left\{\begin{array}{c}\phosphorus[31]+\pt \\ \silicon+\helium \end{array}\right. ;\]
as with $\carbon+\carbon$, the \pt\ and \helium\ combine with other nuclei to produce a group of nuclei clustered about \silicon.

The strong Coulomb barrier inhibits the fusion of nuclei beyond \oxygen; instead, photodissociation reactions such as $\silicon+\gamma \to \magnesium+\helium$ liberate \nt, \pt, and \helium.  These light nuclei then fuse with other heavy nuclei, and the composition gradually becomes composed of nuclei around \iron.  This is \newterm{nuclear statistical equilibrium}: the composition is in the lowest energy state (most bound) for the ambient density and temperature. As a result, there is no further release of nuclear energy possible. The \iron\ core contracts and becomes degenerate; its mass gradually increases from the burning of surrounding material.

\begin{exercisebox}[Dynamical time of evolved stellar core]
At the onset of \oxygen\ burning in a \val{25}{\Msun} star, the central density (Table~\ref{t.burning-timescales}) is $\val{\sci{3.6}{6}}{\grampercc}$ ($\val{\sci{3.6}{9}}{\kilo\gram\,\meter^{-3}}$).  What is the dynamical time of the core?
\end{exercisebox}

The amount of energy available from the fusion of heavy nuclei is low; as a consequence, the time required for the core to deplete the available fuel grows shorter and shorter, with the final stages occurring in a day (column labeled $\tau$ in Table~\ref{t.burning-timescales}). After the ignition of carbon, the core evolves too quickly for the envelope to keep up. Thus the external appearance of the star provides no window into the final days of burning.

\begin{table}[htp]
\caption[Nuclear burning timescales for massive stars]{\label{t.burning-timescales}Nuclear burning timescales for massive stars. Values taken from \citet{Woosley2002The-evolution-a}.}
\begin{tabular}{rrrrr}
\multicolumn{5}{c}{hydrogen}\\
$M_{\mathrm{ZAMS}}$ (\Msun)& $T_{c}$ ($\val{10^{7}}{\K}$) & $\rho_{c}$ ($\grampercc$) & $L$ ($\val{10^{3}}{\Lsun}$) & $\tau$ (Myr)\\
\hline
15 & 3.53 & 5.81 & 28 & 11.1\\
25 & 3.81 & 3.81 & 110 & 6.7\\
\hline
\multicolumn{5}{c}{helium}\\
$M_{\mathrm{ZAMS}}$ (\Msun)& $T_{c}$ ($\val{10^{8}}{\K}$) & $\rho_{c}$ ($\val{10^{3}}{\grampercc}$) & $L$ ($\val{10^{3}}{\Lsun}$) & $\tau$ (Myr)\\
\hline
15 & 1.78 & 1.39 & 41 & 1.97\\
25 & 1.96 & 0.76 & 182 & 0.84\\
\hline
\multicolumn{5}{c}{carbon}\\
$M_{\mathrm{ZAMS}}$ (\Msun)& $T_{c}$ ($\val{10^{8}}{\K}$) & $\rho_{c}$ ($\val{10^{6}}{\grampercc}$) & $L$ ($\val{10^{3}}{\Lsun}$) & $\tau$ (kyr)\\
\hline
15 & 8.34 & 2.39 & 83 & 2.03\\
25 & 8.41 & 1.29 & 245 & 0.52\\
\hline
\multicolumn{5}{c}{oxygen}\\
$M_{\mathrm{ZAMS}}$ (\Msun)& $T_{c}$ ($\val{10^{9}}{\K}$) & $\rho_{c}$ ($\val{10^{6}}{\grampercc}$)& $L$ ($\val{10^{3}}{\Lsun}$) & $\tau$ (yr)\\
\hline
15 & 1.94 & 6.66 & 87 & 2.58\\
25 & 2.09 & 3.60 & 246 & 0.40\\
\hline
\multicolumn{5}{c}{silicon}\\
$M_{\mathrm{ZAMS}}$ (\Msun)& $T_{c}$ ($\val{10^{9}}{\K}$) & $\rho_{c}$ ($\val{10^{7}}{\grampercc}$) & $L$ ($\val{10^{3}}{\Lsun}$) & $\tau$ (d)\\
\hline
15 & 3.34 & 4.26 & 87 & 18.3\\
25 & 3.65 & 3.01 & 246 & 0.7\\
\end{tabular}
\end{table}

\subsection{Core collapse}
When the core of a massive star reaches nuclear statistical equilibrium (NSE), there are no further sources of energy available. Fusion reactions in the shells surrounding the core add mass to it, causing it to contract. The increasing density raises the electron Fermi energy. When the Fermi energy approaches the rest mass of the electrons---$m_{e}c^{2} = \val{0.511}{\MeV}$---the electrons move relativistically. This alters the equation of state.

The reason is that the energy no longer goes as $p^{2}/2m$ for relativistic particles. The correct relation is
\[
	E = \sqrt{p^{2}c^{2} + m^{2}c^{4}} = mc^{2}\sqrt{1 + \left(\frac{p}{mc}\right)^{2}};
\]
when $p/mc\ll 1$, we can expand this as $E\approx m c^{2} + p^{2}/2m$---that is, as the sum of the rest mass and the Newtonian form of the kinetic energy. We'll now explore the opposite limit, with $p \gg mc$, so that $E = pc$.

Recall that for a degenerate gas, we began filling energy states, starting with the lowest open levels until we have added all $N$ electrons (eq.~[\ref{e.number-degenerate}]):
\[
	N = \frac{2}{h^{3}}\int_{V}\dif^{3}x\int_{0}^{\EF}\dif^{3}p.
\]
We then change variables, $\dif^{3}p = 4\pi p^{2}\,\dif p = 4\pi c^{-3}\varepsilon^{2}\,\dif\varepsilon$, where $\varepsilon = pc$ is the energy of a single electron:
\[
	N = \frac{8\pi}{h^{3}c^{3}}V \int_{0}^{\EF} \varepsilon^{2}\,\dif\varepsilon
	= \frac{8\pi}{3h^{3}c^{3}} V  \EF^{3}.
\]
Solving for the Fermi energy,
\[
	\EF = hc\left(\frac{3}{8\pi}\frac{N}{V}\right)^{1/3}.
\]
To get the total energy, we multiply each electron by its energy $\varepsilon$ and integrate over phase space:
\[
	E = \frac{8\pi}{h^{3}c^{3}}V\int_{0}^{\EF}\varepsilon^{3}\,\dif\varepsilon = 
		\frac{1}{4}\frac{8\pi}{h^{3}c^{3}}V\EF^{4} = \frac{3}{4}N\EF.
\]
For a relativistic gas, the pressure is $P = (1/3)(E/V)$ (cf.\ Box~\ref{sb.radiation-pressure}), so that
\begin{equation}
\label{e.pressure-relativistic}
	P = \frac{1}{4}n\EF = \frac{1}{4}\left(\frac{3}{8\pi}\right)^{1/3}hc n^{4/3},
\end{equation}
with $n = \rho/\mu_{e}m_{u}$.

\subsection{The Chandrasekhar mass}

In exercise~\ref{ex.degenerate-mass-radius}, we constructed a mass-radius relation for white dwarfs by combining the virial relations,
\begin{eqnarray*}
   P    &\propto& \frac{GM^{2}}{R^{4}}\\
   \rho &\propto& \frac{M}{R^{3}}
\end{eqnarray*}
and the equation of state for a non-relativistic, degenerate, ideal gas.  We found that $R\propto M^{-1/3}$.  If we try that with our relativistic equation of state, eq.~(\ref{e.pressure-relativistic}), we get
\[
	\frac{GM^{2}}{R^{4}} \propto P = \frac{1}{4}\left(\frac{3}{8\pi}\right)^{1/3}hc\; \left(\frac{\rho}{\mb\mu_{e}}\right)^{4/3} \propto \frac{M^{4/3}}{R^{4}}.
\]
The radius $R$ cancels, and what we have is a relation $M\propto (hc/G)^{3/2}/\mb^{2}$.  This is rather odd: a gas with a relativistic equation of state in hydrostatic balance has a characteristic mass defined in terms of fundamental constants.

Let's investigate this further. Suppose we have a box with adjustable sides, which we pack with $N$ degenerate electrons. We add some nuclei for mass, so that the total mass in the box is $\mu_{e}\mb N$. The volume of the box $V \sim R^{3}$, and since the electrons are degenerate, the volume per electron is roughly $\lambda^{3}$, where $\lambda \sim h/p$ is the wavelength of the electrons.  As a result, $N = (R/\lambda)^{3}$; further, the momentum of an electron is
\[	p \sim \frac{h}{\lambda} \sim h\frac{N^{1/3}}{R}. \]
If our electrons were non-relativistic, the total, kinetic plus gravitational, energy of our box would be
\[
	E_{\mathrm{total}} = N\frac{p^{2}}{2m_{e}} - \frac{GM^{2}}{R} \sim N^{5/3}\frac{h^{2}}{R^{2}m_{e}} - GN^{2}\mu_{e}^{2}\mb^{2} \frac{1}{R}.
\]
For a given $N$, we can adjust $R$ to make $E_{\mathrm{total}}<0$, and indeed, if we satisfy the virial theorem, we will recover the $R\propto M^{-1/3}$ scaling.

If, however, the electrons are relativistic then the total energy is
\begin{eqnarray*}
	E_{\mathrm{total}} = Npc - \frac{GM^{2}}{R} 
		&=& \frac{1}{R}\left[hc N^{4/3} - G N^{2}(\mu_{e}\mb)^{2}\right]\\
		&=& G(\mu_{e}\mb)^{2}\frac{N^{4/3}}{R}
		{\color{red}\left[ \frac{hc}{G(\mu_{e}\mb)^{2}} - N^{2/3}\right]}.
\end{eqnarray*}
Look at the term in $\color{red}\left[\cdot\right]$.
If $N < [hc/G/(\mu_{e}\mb)^{2}]^{3/2}$, then $E_{\mathrm{total}} > 0$; by making $R$ larger, however, we can lower the energy until the electrons are no longer relativistic.  If $N > [hc/G(\mu_{e}\mb)^{2}]^{3/2}$, then $E_{\mathrm{total}} < 0$; by making $R$ smaller, however, we can keep reducing $E_{\mathrm{total}}$ indefinitely.
\begin{quote}\itshape
There is no bound state with finite $R$ for $M>(hc/G)^{3/2}(\mu_{e}\mb)^{-2}$.
\end{quote}

\begin{sidebar}[Instability for a relativistic equation of state]
There is another way of looking at the onset of instability which is instructive (this treatment follows that in \citet{Cox1980Theory-of-Stell}). In exercise \ref{ex.stellar-oscillation-period} you found that during a contraction or expansion, the equation of motion for a thin layer at the star's surface was
\[
	\ddot{\delta R} = \frac{GM}{R^{2}}\left[4-3\gamma\right]\frac{\delta R}{R}.
\]
Here $M$ and $R$ are the total stellar mass and radius, and the adiabatic pressure-density relation is $P\propto \rho^{\gamma}$.

For a non-relativistic gas, $\gamma = 5/3$, and so $\ddot{\delta R} \propto -\delta R$: the star oscillates with a period that is comparable to the dynamical timescale of the star. If, however, $\gamma < 4/3$, the the equation of motion is $\ddot{\delta R} \propto \delta R$, which has an exponential solution: squeeze the star slightly, and it will collapse!

Let's work out a more physical explanation for what is happening. Suppose we have a star in virial equilibrium. Then the central pressure and density are
\begin{eqnarray*}
P &\propto& \frac{GM^{2}}{R^{4}} \\
\rho &\propto& \frac{M}{R^{3}}.
\end{eqnarray*}
Now if we contract the star by a small amount, say $\delta R/R = -1\%$, then the density increases, $\delta\rho/\rho = -3\delta R/R = 3\%$. How does the pressure respond? If the star contracts on a slowly, on a Kelvin-Helmholtz timescale, then there is time for heat to radiate away, so that the internal pressure can increase by the amount needed to maintain equilibrium, $\delta P/P = -4\delta R/R = 4\%$. Under an \emph{adiabatic} contraction, however, there is not enough time for the star to radiate away excess heat; as a consequence, the pressure and density are linked, $\delta P/P = \gamma\delta \rho/\rho = -3\gamma\delta R/R$.

If $\gamma = 4/3$, then during an adiabatic compression of $\delta R/R = -1\%$, the density increases by $3|\delta R/R| = 3\%$ and the pressure increases by $3\gamma|\delta R/R| = 4\%$, which is precisely the increase needed to maintain mechanical equilibrium. As a result, the star remains in hydrostatic balance at its new, smaller radius. This is why there was no mass-radius relation for $\gamma = 4/3$; it takes no energy to contract (or expand) the star.

For $\gamma > 4/3$, when the star contracts the central pressure increases by $3\gamma|\delta R/R| > 4|\delta R/R|$. As a result, the pressure becomes greater than the amount needed for hydrostatic balance. This excess pressure pushes the star outward and acts as a restoring source. During an expansion, the pressure falls below the amount needed for hydrostatic equilibrium, so gravity halts the expansion and forces the star to contract. Hence, for $\gamma > 4/3$, the star responds to a radial perturbation by oscillating with a period comparable to the dynamical timescale (cf.\ exercise \ref{ex.stellar-oscillation-period}).

In contrast, for $\gamma < 4/3$, the increase in pressure during contraction is $3\gamma|\delta R/R| < 4|\delta R/R|$. The gas pressure does not increase enough to supply the pressure needed to maintain hydrostatic equilibrium, and so the star's contraction accelerates. A small perturbation inwards leads to implosion.
\end{sidebar}

Thus, there is a limit to the total mass that can be supported in hydrostatic equilibrium by degenerate electrons. 
An exact calculation for the maximum mass of a cold star yields
\begin{equation}\label{e.Chandrasekhar}
	M_{\mathrm{Ch}} = 1.456 \left(\frac{2}{\mu_{e}}\right)^{2}\Msun.
\end{equation}
When the mass reaches this limiting value, known as the \newterm{Chandrasekhar mass}\sidenote{Derived by S. Chandrasekhar at age 20(!) while traveling from India to England in 1930}, the electrons become relativistic and $\partial P/\partial \rho \to 4/3$; the star becomes unstable and collapses.

When the core of a massive star begins its collapse, the electron Fermi is $\sim\MeV$, which is sufficient to induce electron captures on iron-group nuclei. These captures increase $\mu_{e}$ and reduce $M_{\mathrm{Ch}}$. As the core begins the final plunge, the rapidly rising temperature induces the photodissociation of iron-group nuclei into neutrons, protons, and helium nuclei. This process is endothermic, which further robs the core of pressure support and accelerates the collapse. The effective $\gamma = \partial P/\partial\rho < 4/3$ on account of the photodissociation and electron captures, and the core implodes.

When the core density approaches $\val{0.16}{\fermi^{-3}}$, the nucleons begin to repel one another on account of the strong nuclear force. At this point the collapse halts, sending a shockwave outwards. The core now consists mostly of neutrons and is termed a \newterm{neutron star}.

\begin{exercisebox}[Gravitational binding energy of a neutron star]
What is the mass density if the number density of nucleons is $\val{0.16}{\fermi^{-3}}$? What is the gravitational binding energy for an object with a mass \val{1.4}{\Msun} at this density?
\end{exercisebox}

The outward going shockwave soon stalls as the outer layers of the star fall inward. The energy needed to blow the envelope off is about 1\% of the gravitational binding energy of the core, so there is plenty of energy available to disperse the envelope if this energy can be tapped. Most of the gravitational binding energy released by the imploding core is carried outwards by neutrinos. 
During the collapse, the neutrino mean free path becomes smaller than the core radius for two reasons: the weak interaction cross-section increases as the nucleons reach temperatures $\gtrsim\val{10^{10}}{\K}$, and the mean free path $\ell = (n\sigma)^{-1}$ decreases with density. 
As a result the neutrinos become trapped and must diffuse of the collapsing core. As the neutrinos diffuse out, they transfer a small fraction of their energy to the material heating it. This tends to push the shock outward. A competition arises between the ram pressure of infalling matter and the heating from the neutrinos. If the neutrinos can transfer enough energy to the envelope, then the envelope will be blown off in a supernova. If not, then matter will continue to accumulate onto the neutron star. The maximum mass of a neutron star is uncertain, but on physical grounds is likely $< \val{3}{\Msun}$. If the shock is not re-energized, then conceivably the entire star could implode into a black hole.

\section{Stellar resurrection}
\label{s.stellar-resurrection}

In the previous section, we learned that stars with $M\lesssim\valrng[--]{8}{10}{\Msun}$ eventually become white dwarfs supported by electron degeneracy pressure and composed of carbon and oxygen nuclei; and that more massive stars have cores that either collapse to form neutron stars supported by the strong nuclear interaction or collapse fully into black holes.


