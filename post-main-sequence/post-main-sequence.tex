% !TEX root = ../intro-stellar-physics.tex

\section{Low-mass stars}
For stars with main-sequence masses $\lesssim \valrng[--]{8}{10}{\Msun}$, the core becomes degenerate before the onset of $\carbon$ fusion.  Indeed, for stars around a solar mass, the fusion of \helium\ occurs under moderately degenerate conditions.

\subsection{Ascent of the red-giant branch}
With the depletion of hydrogen, the core contracts while there is \hydrogen\ fusion occurring in a shell surrounding the core.  The rising temperature and pressure in the shell causes the \hydrogen\ fusion reactions to go at an ever increasing rate.  The high luminosity drives the envelope, now fully convective, to large radii and hence to a low surface temperature. Observationally, the star becomes a red giant.  During this phase the high luminosity, combined with the low surface gravity of the distended envelope, drives a strong wind\sidenote{The exact rate of mass loss is still not well constrained.}.

\subsection{The helium flash}
There are no stable isotopes with mass $A=5$ or $A=8$, which makes the fusion of \helium\ somewhat tricky.  The isotope \beryllium[8] is, however, long-lived ($\val{10^{-16}}{\second}$) compared to a nuclear timescale\sidenote{Roughly the time for a pion to cross a nucleus, $\sim \val{10^{-22}}{\second}$.}.
When the core temperature reaches $\approx\val{10^{8}}{\K}$, the reaction
\[ \helium + \helium \longleftrightarrow\beryllium[8] \]
builds up a small abundance of \beryllium[8]; the reaction
\[ \beryllium[8] + \helium \longleftrightarrow \carbon^{*} \]
then builds up a small abundance of \carbon\ in an excited state (denoted by the $^{*}$).  While the dominant decay channel of $\carbon^{*}$ is $\carbon^{*}\to\beryllium[8]+\helium$, there is a branching to the ground state, $\carbon^{*}\to\carbon+\gamma$; as a result, there is a net conversion $3\helium\to\carbon$---the ``triple-alpha'' reaction.  This reaction is \emph{very temperature-sensitive}: $\partial\ln\epsilon_{3\alpha}/\partial\ln T \approx 40$ at $T = \val{10^{8}}{\K}$; this sensitivity, combined with the mildly degenerate conditions of the core, makes the ignition of \helium\ unstable for solar-mass stars.

\subsection{Helium burning: the horizontal branch}

Once core \helium\ has ignited, the star settles onto a ``Helium Main Sequence;'' observationally this is the \newterm{horizontal branch}, so called because these stars form a noticeable clump on a Hertzsprung-Russell diagram.  The luminosity on the horizontal branch is about \valrng[--]{30}{100}{\Lsun}.  The higher luminosity, combined with the much lower energy release from the $3\helium\to\carbon$, makes the horizontal branch lifetime much shorter than that of the main-sequence (e.g., the horizontal branch lifetime is $\sim \val{10^{8}}{\yr}$ for a solar-mass star.

\subsection{The asymptotic giant branch and planetary nebula}

With the depletion of \helium\ in the core, the core---now composed of \carbon\ and \oxygen---contracts, while the growing luminosity from the H- and He-burning shells again drive the envelope to large radii.  The H- and He-burning shells are often thermally unstable and \emph{pulse}.  The hydrogen-rich envelope is consumed at its base by the H- and He-burning shells and is expelled at the surface by an increasingly strong wind.  After the envelope is gone, the hot core---now called a \newterm{white dwarf}---slowly cools.

\section{Massive stars}

For stars with main-sequence masses $\gtrsim \valrng[--]{8}{10}{\Msun}$, the fusion of \carbon\ commences while the core is non-degenerate.  The temperature required is $\approx\val{\sci{8}{8}}{\K}$.  At this temperature, electron-positron pairs can form and annihilate; while this usually produces photons, there is a branch
\[ e^{-}+e^{+} \longleftrightarrow \nu_{e} + \bar{\nu}_{e}\]
that generates a neutrino-antineutrino pair. The mean free path for the neutrinos is larger than the radius of the star; as a result, this process efficiently cools the core.

After this point, the core evolves on a timescale that is too fast for the envelope to keep up.  The fusion of \carbon\ produces $\sodium+\pt$ and $\neon+\helium$; the \pt\ and \helium\ capture onto other nuclei that are present.  At slightly higher temperatures, the reaction $\neon+\gamma \to \oxygen+\helium$ occurs; the \helium\ capture onto other \oxygen, \neon, and \magnesium.

The next significant burning stage is $\oxygen+\oxygen$, which produces $\phosphorus[31]+\pt$ and $\silicon+\helium$; as before, the \pt\ and \helium\ combine with other nuclei and produce a group of nuclei around \silicon.

The strong Coulomb barrier inhibits the fusion of nuclei beyond \oxygen; instead, photodissociation reactions such as $\silicon+\gamma \to \magnesium+\helium$ liberate \nt, \pt, and \helium.  These light nuclei then fuse with other heavy nuclei, and the composition gradually becomes composed of nuclei around \iron.  This is \emph{nuclear statistical equilibrium}: the composition is in the lowest energy state (most bound) for the ambient density and temperature.

\begin{exercisebox}[Dynamical time of evolved stellar core]
At the onset of \oxygen\ burning in a \val{20}{\Msun} star, the central density is $\approx \val{10^{6}}{\grampercc}$.  What is the dynamical time of the core?
\end{exercisebox}

\subsection{A relativistic degenerate equation of state}

For degenerate electrons, the momentum---and hence the kinetic energy---increases with density.  When the kinetic energy approaches the rest mass of the electrons, the relation between kinetic energy and momentum is no longer $E = p^{2}/2m$.  When the electrons are extremely relativistic, their energy becomes $E = p c$.

Recall that for an ideal gas (no interactions) the density of electrons is given by the integral over phase space,
\[
  n = \frac{2}{h^{3}} \int \dif^{3}p \frac{1}{\exp\left[\left(E-\mu\right)/kT\right]+1}.
\]
The $2$ in the numerator is because electrons are spin-$1/2$ particles and there are thus 2 states in the same infinitesimal $\dif^{3}p$.  If the electrons are degenerate, $kT \ll E,\mu$ and the integral becomes
\[
  n = \frac{2}{h^{3}}\int_{0}^{E_{F}} \dif^{3}p = \frac{8\pi}{h^{3}} \int_{0}^{E_{F}} p^{2}\,\dif p = \frac{8\pi}{h^{3}c^{3}} \int_{0}^{E_{F}} E^{2}\,\dif E = \frac{8\pi}{3h^{3}c^{3}} E_{F}^{3}.
\]
Hence the electrons are characterized by a Fermi energy
\begin{equation}\label{e.fermi}
	E_{F} = \left(\frac{3}{8\pi}\right)^{1/3}hc\; n^{1/3}
\end{equation}
With this we can find the energy density
\[
  U = \frac{8\pi}{h^{3}c^{3}}\int_{0}^{E_{F}} E^{3}\,\dif E = \frac{3}{4} n E_{F};
\]
and for a relativistic gas the pressure is $U/3$, or
\begin{equation}\label{e.pressure-relativistic}
  P = \frac{1}{4}n E_{F} = \frac{1}{4}\left(\frac{3}{8\pi}\right)^{1/3}hc\; n^{4/3}.
\end{equation}
Here $n = \rho/(\mb\mu_{e})$ is the number of electrons per unit volume.

\subsection{The Chandrasekhar mass}
We constructed a mass-radius relation for white dwarfs by combining the virial relations,
\begin{eqnarray*}
   P    &\propto& \frac{GM^{2}}{R^{4}}\\
   \rho &\propto& \frac{M}{R^{3}}
\end{eqnarray*}
and the equation of state for a non-relativistic, degenerate, ideal gas.  We found that $R\propto M^{-1/3}$.  If we try that with our relativistic equation of state, eq.~(\ref{e.pressure-relativistic}), we get
\[
	\frac{GM^{2}}{R^{4}} \propto P = \frac{1}{4}\left(\frac{3}{8\pi}\right)^{1/3}hc\; \left(\frac{\rho}{\mb\mu_{e}}\right)^{4/3} \propto \left(\frac{M}{R^{3}}\right)^{4/3}.
\]
The radius $R$ cancels, and what we have is a relation $M\propto (hc/G)^{3/2}/\mb^{2}$.  This is rather odd: hydrostatic balance and our equation of state produce a characteristic mass in terms of fundamental constants.

Let's investigate this a bit more.  Suppose we construct a box filled with $N$ electrons and a mass $\mu_{e}\mb N$.  The volume of the box $V \sim R^{3}$, and since the electrons are degenerate, the volume per electron is roughly $\lambda^{3}$, where $\lambda \sim h/p$ is the wavelength of the electrons.  As a result, $N = (R/\lambda)^{3}$; further, the momentum of an electron is
\[	p \sim \frac{h}{\lambda} \sim h\frac{N^{1/3}}{R}. \]
If our electrons were non-relativistic, the total, kinetic plus gravitational, energy of our box would be
\[
	E_{\mathrm{total}} = N\frac{p^{2}}{2m_{e}} - \frac{GM^{2}}{R} \sim N^{5/3}\frac{h^{2}}{R^{2}m_{e}} - GN^{2}\mu_{e}^{2}\mb^{2} \frac{1}{R}.
\]
For a given $N$, we can adjust $R$ to make $E_{\mathrm{total}}<0$, and indeed, if we satisfy the virial theorem, we will recover the $R\propto M^{-1/3}$ scaling.

If, however, the electrons are relativistic then the total energy is
\begin{eqnarray*}
	E_{\mathrm{total}} = Npc - \frac{GM^{2}}{R} 
		&=& \frac{1}{R}\left[hc N^{4/3} - G N^{2}(\mu_{e}\mb)^{2}\right]\\
		&=& G(\mu_{e}\mb)^{2}\frac{N^{4/3}}{R}\left[ {\color{red}\frac{hc}{G(\mu_{e}\mb)^{2}} - N^{2/3}}\right].
\end{eqnarray*}
If $N < [hc/G/(\mu_{e}\mb)^{2}]^{3/2}$, then $E_{\mathrm{total}} > 0$; by making $R$ larger, however, we can lower the energy until the electrons are no longer relativistic.  If $N > [hc/G(\mu_{e}\mb)^{2}]^{3/2}$, then $E_{\mathrm{total}} < 0$; by making $R$ smaller, however, we can keep reducing $E_{\mathrm{total}}$.  There is no bound state with finite $R$.

Thus, there is a limit to the number of degenerate electrons, and hence the total mass, that we can put in hydrostatic equilibrium.  An exact calculation yields
\begin{equation}\label{e.Chandrasekhar}
	M_{\mathrm{Ch}} = 1.456 \left(\frac{2}{\mu_{e}}\right)^{2}\Msun.
\end{equation}
When the mass reaches this limiting value, known as the \newterm{Chandrasekhar mass}, the star must collapse.  

\newthought{There is another way of looking at the onset of instability}\cite{Cox1980Theory-of-Stell} that touches on a broader topic in physics.  Suppose we imagine all of the mass of the star $M$ is concentrated at the center, and the volume of the star is filled with a massless gas of mean density $n = N/V = 3N/(4\pi R^{3})$. Encasing the star is an impermeable membrane of mass $m$.  The net force on the membrane is
\begin{equation}\label{e.force}
	4\pi R^{2} P - \frac{GMm}{R^{2}} = 0,
\end{equation}
and we set this net force to zero because we presume the membrane is in equilibrium at radius $R$.

Now let us change the radius by a small amount $\delta R$ and get the equation of motion. Our perturbation is adiabatic, in which case the new pressure is
\[ 
	P(R+\delta R) \approx P + \tderiv{P}{n}{s}\DD{n}{R}\delta R 
	= P + \frac{P}{n}\tderiv{\ln P}{\ln n}{s} \DD{n}{R}\delta R. 
\]
To make this simple, write $n^{-1}\dif n/\dif R = R^{-1}\dif\ln n/\dif\ln R = -3/R$; using this and the adiabatic exponent $\gamma = (\partial\ln P/\partial\ln n)_{s}$ gives
\[
	P(R+\delta R) \approx P\left(1 + \frac{\delta P}{P}\right) 
	= P\left(1 - 3\gamma \frac{\delta R}{R}\right).
\]
To lowest order the area term is
\[ 4\pi (R+\delta R)^{2} \approx 4\pi R^{2}\left(1 + 2\frac{\delta R}{R}\right), \]
and the gravitational term is
\[
	\frac{GMm}{(R+\delta R)^{2}} \approx \frac{GMm}{R^{2}}\left(1-2\frac{\delta R}{R}\right).
\]
Now we substitute these expressions into equation (\ref{e.force}) and keep only terms to linear order in $\delta R$:
\begin{eqnarray}
	F &=& \color{red} 4\pi R^{2} P - \frac{GMm}{R^{2}}
	\color{black} + \left[{\color{red}4\pi R^{2}P}\left( 2 - 3\gamma \right) + 2 {\color{red}\frac{GMm}{R^{2}}} \right ] \frac{\delta R}{R} \nonumber\\
	&=& \frac{GMm}{R^{2}} \left[4 - 3\gamma\right]\frac{\delta R}{R}.
\end{eqnarray}
The terms in red are equal, because we assumed when $\delta R = 0$ the force vanishes.  Notice the sign of the force: if $\gamma > 4/3$, the force has an opposite sign to $\delta R$. For example, if $\gamma = 5/3$, then 
\begin{equation}\label{e.SHO}
	m\frac{\dif^{2}\delta R}{\dif t^{2}} = F  = - \frac{GMm}{R^{3}}\delta R.
\end{equation}
This is just the equation of motion of a simple harmonic oscillator.

\begin{exercisebox}[Stellar pulsation period]
What is the period of oscillation for equation (\ref{e.SHO})?  Notice anything familiar about this timescale?
\end{exercisebox}

\noindent If, however, $\gamma < 4/3$, the the equation of motion is
\[ \delta\ddot{R} \propto \delta R, \]
which has an exponential solution: push in the membrane, and it continues to collapse!

\subsection{Core collapse}
When the core of a massive star reaches nuclear statistical equilibrium (NSE), there are no further sources of energy available. Fusion reactions in the shells surrounding the core add mass to it, causing it to contract.  The increasing density raises the electron Fermi energy.  At some point, the electron Fermi is sufficiently large $\sim\MeV$ that electron captures on iron-group nuclei commence.  These captures do two things: first, by removing some of the electrons they force the remaining ones to have a higher momentum and energy to maintain pressure support; second, the captures increase $\mu_{e}$ and reduce $M_{\mathrm{Ch}}$.  As the core begins the final plunge, the rapidly rising temperature induces the photodisintegration of iron-group nuclei into neutrons, protons, and helium nuclei.  This process is endothermic, which further robs the core of pressure support and accelerates the collapse.

As the core collapses, electrons combine with protons to form neutrons; in the process, a large number of neutrinos are released.  Eventually the neutrino mean free path becomes smaller than the core radius for two reasons: the weak interaction cross-section increases with energy, and the mean free path increases with density, $\ell = (n\sigma)^{-1}$.  At this point neutrinos must diffuse out of the collapsing core.

The gravitational binding energy released during collapse is enormous,
\[ E_{\mathrm{grav}} \sim \frac{GM^{2}}{R} \sim \val{10^{53}}{\erg}. \]
Most of this is carried off in neutrinos; if a small amount $\sim \val{10^{51}}{\erg}$ can be transferred to the envelope, then the envelope will be blown off producing a \newterm{supernovae}.

