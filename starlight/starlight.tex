% !TEX root = ../intro-stellar-physics.tex

\section{Introduction: Our Sun}

Let's start by considering the star we know best: our sun.  We'll denote the sun with the symbol $\sun$.  From the orbits of the planets we can deduce the mass of the sun from Kepler's laws:
\[
	\Msun = \val{\sci{1.99}{30}}{\kilo\gram}.
\]
Radar ranging of the solar system gives us the mean Earth-Sun distance, which defines the \emph{astronomical unit}
\[
	\val{1}{\AU} = \val{\sci{1.5}{11}}{\meter}.
\]
Knowing this distance and the angular size of the sun then tells us its radius,
\[
	\Rsun = \val{\sci{6.96}{8}}{\meter}.
\]
From measurements of the radiant flux and the distance, we then can infer the sun's radiant power, or luminosity,
\[
	\Lsun = \val{\sci{3.86}{27}}{\watt}.
\]

\begin{exercisebox}[Solar power]
Suppose we wish to replace the Simon power plant with a grid of solar panels. Under ideal conditions (direct light and 100\% efficient panels), how many square meters of solar panels are needed to generate $\val{70}{\Mega\watt}$ ($\val{\sci{70}{6}}{\watt}$)?
\end{exercisebox}
%The planetary orbits and the gravitational constant $G$ tell us its mass; our knowledge of the earth-sun distance and observations tell us its radius; measurements of the solar radiant flux and spectra tell us its luminosity and temperature; and radiometric dating of meteorites tells us the age of the solar system. In summary:
%\begin{eqnarray*}
%\Msun &=& 1.99\ee{33}\nsp\gram\\
%\Rsun &=& 6.96\ee{10}\nsp\cm\\
%\Lsun &=& 3.86\ee{33}\nsp\ergspersecond\\
%\Teff &=& 5780\nsp\K\\
%\tau_{\sun} &=& 4.6\nsp\Giga\yr.
%\end{eqnarray*}
%Moreover, the composition of the sun is well known\cite{anders.grevesse:abundances,Asplund2005The-Solar-Chemi}; the five most abundant elements are H, He($-1.07$), N($-4.22$), O($-3.34$), and C($-3.61$), where the number in parentheses is $\log(n_{\mathrm{el}}/n_{\mathrm{H}})$, the abundance relative to hydrogen.