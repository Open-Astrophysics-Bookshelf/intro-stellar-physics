% !TEX root = ../intro-stellar-physics.tex

\section*{Preface}
These notes were compiled when I taught the junior/senior undergraduate course on stars at Michigan State University in the autumns of 2012, 2014, and 2016, and were typeset, with additional figures and exercises, during sprint and summer of 2018. The motivation for typesetting the notes was twofold: first, I was already providing many supplemental handouts and exercises for the students, so it made sense to compile them; second, the reduced reliance on a textbook made it difficult to justify the additional cost.

The text layout uses the \code{tufte-book} (\url{https://tufte-latex.github.io/tufte-latex/}) \LaTeX\ class: the main feature is a large outer margin in which the students can take notes; this margin also holds small figures and footnotes. Exercises are embedded throughout the text. The exercises range from comprehension checks to longer, more challenging problems.  Because the exercises are spread throughout the text, there is a ``List of Exercises'' in the front matter to help with looking for specific problems. I've also added boxes containing more advanced material that I felt students should be exposed to, but were not essential to the main development of the course. The design is meant to encourage students to actively work through the notes, and it will be interesting to see if that in fact occurs.

In designing the course, I also wanted the students to gain competency in ``thinking like a physicist:'' by which I mean, using physical concepts to analyze the behavior of macroscopic objects, such as stars. Many of the problems, therefore, make use of simplifed models or dimensional analysis rather than elaborate calculation. 

To liven up the chapter titles, I used song names. The first two were easy; my rules for the following chapters then were that the song had to make sense, both topically and musically. By musically I mean that when adding a song title, it had to be in a similar style with the ones already in the list. The credits, in order, go to Muse; Queen/David Bowie; Greta van Fleek; Dio; Deep Purple; David Bowie; and the Traveling Wilburys.

\newthought{Please be advised that these notes are under active development;} to refer to a specific version, use the eight-character stamp labeled ``git version'' on the copyright page.
