% !TEX root = ../intro-stellar-physics.tex

\section*{Preface}
These notes were gradually written as I taught the junior/senior undergraduate course on stars at Michigan State University in the autumn semesters of 2012, 2014, and 2016. They were finally put into manuscript form, with additional figures and exercises, during the spring and summer of 2018. Further notes on numerical methods were added in the spring and summer of 2020. The motivation for assembling the notes was to make a self-contained package that could be inexpensively distributed to students instead of a textbook.

In addition to deriving a basic physical description of how stars work, a secondary goal of the course is to train students to make simple physical models and order-of-magnitude estimates. This is a crucial skill that is not incorporated enough into the typical undergraduate physics courses. In keeping with this goal, many of the exercises ask the students to make estimates or to employ simple models, such as constant density throughout the star, rather than to perform elaborate calculations. There are some exercises in the text that must be solved numerically, and the course does include a group numerical project. I've therefore appended some notes on common (in stellar physics) numerical tasks: finding roots, solving ordinary differential equations, and interpolating tabulated data. There are a number of excellent references and numerical libraries available for these tasks; the goal of the appendix is to introduce the technique and explain its basic workings.

\newthought{In laying out the course outline, there were several options} for the order in which to present material. One would be to start with chapter~\ref{ch.basic-stellar-properties}, which covers hydrostatic equilibrium and establishes estimates for the mean stellar density, pressure, and temperature. The material in chapter~\ref{ch.starlight} on radiant intensity, flux, and thermal emission would then be introduced in chapters~\ref{ch.radiative-transport} and \ref{ch.classifying-stars}, which cover radiative heat transport in the stellar interior and the conditions at the photosphere. The remainder of chapter~\ref{ch.starlight} on magnitudes would then come later, perhaps in chapter~\ref{ch.main-sequence} where we discuss the main sequence.

Although this order is logical---and one an instructor could reasonably choose---I decided on the layout used here for pedagogical reasons. First, radiative transfer is a difficult subject, and introducing the basic concepts early gives the students more time to become familiar with the topic. Second, finding Wien's law requires a numerical rootfind (exercise~\ref{ex.Wien-wavelength}), and this is a good warmup for further numerical projects. Finally, the discussion of radiative intensity allows us to introduce magnitudes and color indices, and it is good to establish contact with subject's observational foundations at the start.

\newthought{The text layout uses the \code{tufte-book} \LaTeX\ class}\sidenote{\url{https://tufte-latex.github.io/tufte-latex/}}: the main features are a large outer margin in which the students can take notes and the tight integration of text, figures, and sidenotes. Exercises are embedded throughout the text. The exercises range from comprehension checks to longer, more challenging problems. Some of the exercises have a numerical component, denoted with a ``\notebook'' symbol. Because the exercises are spread throughout the text, there is a ``List of Exercises'' in the front. I've also added boxes containing more advanced material that I felt students should be exposed to, but were not essential to the main development of the course. 

\newthought{One evening I tried to liven up the chapter titles.} I noticed that the first two chapters had titles that were also titles for pop songs. I then decided to find song titles that would fit for the remaining chapters. When selecting titles, I imposed a rule that they all could plausibly go together on a playlist. This was challenging since the chapters originally had titles such as ``The equation of state'' and ``The radiative opacity''. The credits for the chapter titles, in order, go to Muse, Queen/David Bowie, Greta van Fleet, Dio, Deep Purple, David Bowie, The Traveling Wilburys, and Muse.

\newthought{Please be advised that these notes are under active development;} to refer to a specific version, use the eight-character stamp labeled ``git version'' on the copyright page.
