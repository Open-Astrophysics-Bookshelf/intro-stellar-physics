% !TEX root = ../intro-stellar-physics.tex

\section*{Preface}
These notes were written as I taught the junior/senior undergraduate course on stars at Michigan State University in the autumn semesters of 2012, 2014, and 2016, and were finally put into manuscript form, with additional figures and exercises, during the spring and summer of 2018. Further notes on numerical methods were added in the Spring and Summer of 2020. The motivation for assembling the notes was to make a self-contained package that could be inexpensively distributed to students instead of a textbook.

In addition to deriving a basic physical description of how stars work, a secondary goal of the course is to train students to make simple physical models and order-of-magnitude estimates. This is a crucial skill that is not incorporated enough into the typical undergraduate physics courses. In keeping with this goal, many of the exercises ask the students to make estimates or to employ simple models, such as constant density throughout the star, rather than to perform elaborate calculations. There are some exercises in the text that must be solved numerically, and the course does include a group numerical project. I've therefore appended some notes on basic, common (in stellar physics) numerical tasks: finding roots, solving ordinary differential equations, and interpolating tabulated data. There are a number of excellent references and numerical libraries available for these tasks; the goal of the appendix is to introduce the technique and explain its basic workings.

\newthought{The text layout uses the \code{tufte-book} \LaTeX\ class}\sidenote{\url{https://tufte-latex.github.io/tufte-latex/}}: the main features are a large outer margin in which the students can take notes and the tight integration of text, figures, and sidenotes. Exercises are embedded throughout the text. The exercises range from comprehension checks to longer, more challenging problems. This layout is meant to encourage students to actively work through the notes, and it will be interesting to see if that in fact occurs. Because the exercises are spread throughout the text, there is a ``List of Exercises'' in the front. I've also added boxes containing more advanced material that I felt students should be exposed to, but were not essential to the main development of the course. 

One evening I tried to liven up the chapter titles. I noticed that the first two chapters had titles that were also titles for pop songs. I then decided to find song titles that would fit for the remaining chapters. When selecting titles, I imposed a rule that they all could plausibly go together on a playlist. This was challenging since the chapters originally had titles such as ``The equation of state'' and ``The radiative opacity''. The credits for the chapter titles, in order, go to Muse, Queen/David Bowie, Greta van Fleet, Dio, Deep Purple, David Bowie, The Traveling Wilburys, and Muse.

\newthought{Please be advised that these notes are under active development;} to refer to a specific version, use the eight-character stamp labeled ``git version'' on the copyright page.
