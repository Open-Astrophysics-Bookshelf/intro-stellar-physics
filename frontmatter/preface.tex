% !TEX root = ../intro-stellar-physics.tex

\section*{Preface}
These notes were collected when I taught the junior/senior undergraduate course on stars at Michigan State University in the autumns of 2012, 2014, and 2016, and were finally written into a manuscript, with additional figures and exercises, during spring and summer of 2018. The motivation for doing so was twofold: first, once I had many supplemental handouts and exercises for the students, it made sense to compile them; second, the high cost of a (at this point, lightly used) textbook makes the cheaper distribution of notes an attractive option.

As I was developing the course, a secondary goal emerged, which was to train students to ``think like a physicist.'' By this I mean, training the students to apply physical concepts to analyze the behavior of macroscopic objects, such as stars. Here the goal is less about using elaborate calculations to reproduce observed behavior, but rather more about making simplified models (or even dimensional analysis) to gain understanding.

The text layout uses the \code{tufte-book} (\url{https://tufte-latex.github.io/tufte-latex/}) \LaTeX\ class: the main feature is a large outer margin in which the students can take notes; this margin also holds small figures and footnotes. Exercises are embedded throughout the text. The exercises range from comprehension checks to longer, more challenging problems. This layout is meant to encourage students to actively work through the notes, and it will be interesting to see if that in fact occurs. Because the exercises are spread throughout the text, there is a ``List of Exercises'' in the front. I've also added boxes containing more advanced material that I felt students should be exposed to, but were not essential to the main development of the course. 

To liven up the chapter titles, I used song names. The first two were easy; in selecting titles for the other chapters, I required that the new song title had to have a similar style with the ones already in the list. The song title credits, in order, go to Muse, Queen/David Bowie, Greta van Fleet, Dio, Deep Purple, David Bowie, and the Traveling Wilburys.

\newthought{Please be advised that these notes are under active development;} to refer to a specific version, use the eight-character stamp labeled ``git version'' on the copyright page.
