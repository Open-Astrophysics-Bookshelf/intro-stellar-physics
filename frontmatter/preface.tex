% !TEX root = ../intro-stellar-physics.tex

\section*{Preface}
These notes were written while teaching an undergraduate-level astronomy course on stars at Michigan State University.  The only background preparation for this course is undergraduate physics at the entering junior level, so there is attention paid to developing topics that undergraduates may not have seen at the time of taking the course.

The text layout uses the \code{tufte-book} (\url{https://tufte-latex.github.io/tufte-latex/}) \LaTeX\ class:  the main feature is a large right margin in which the students can take notes; this margin also holds small figures and sidenotes. Exercises are embedded throughout the text.  These range from ``reading exercises'' to longer, more challenging problems.  Because the exercises are spread throughout the text, there is a ``List of Exercises'' in the front matter to help with looking for specific problems.

The course notes were originally meant as a supplement to the main text.  These notes therefore tend to expand upon topics not already covered there.  In the second half of the course, the students worked on a numerical project.

Some of the material was inspired by three courses at UC-Berkeley in the mid-90's: ``Stars with Lars'', taught by Professor L. Bildsten; Statistical Physics, taught by Professor E.~Commins, and Fluid Mechanics, taught by Professor J. Graham.  I am also indebted to the students who took the MSU stellar physics course for their questions, feedback, and encouragement. 

\newthought{Please be advised that these notes are under active development;} to refer to a specific version, use the eight-character stamp labeled ``git version'' on the copyright page.
