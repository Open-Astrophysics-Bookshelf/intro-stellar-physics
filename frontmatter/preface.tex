% !TEX root = ../intro-stellar-physics.tex

\section*{Preface}
These notes were written while teaching an undergraduate-level astronomy course on stars at Michigan State University. The course was offered in alternating years, so the students were a mix of juniors and seniors, and the only required background preparation for this course were undergraduate physics at the freshman and sophomore level. As a result, I've attempted to develop topics that undergraduates may not have seen at the time of taking the course, such as the virial theorem.

The text layout uses the \code{tufte-book} (\url{https://tufte-latex.github.io/tufte-latex/}) \LaTeX\ class:  the main feature is a large outer margin in which the students can take notes; this margin also holds small figures and sidenotes. Exercises are embedded throughout the text.  These range from ``reading exercises'' to longer, more challenging problems.  Because the exercises are spread throughout the text, there is a ``List of Exercises'' in the front matter to help with looking for specific problems. I've also added boxes containing more advanced material that I felt students should be exposed to, but were not essential to the main development of the course.

In the second half of the course, the students worked on a numerical project.

To liven up the chapter titles, I used song names. Some were obvious fits, some took a bit of searching. Muse, Queen/David Bowie, Earth, Wind, and Fire, Greta van Fleek, Coldplay, Bob Marley, Deep Purple, David Bowie, Traveling Wilburys.

\newthought{Please be advised that these notes are under active development;} to refer to a specific version, use the eight-character stamp labeled ``git version'' on the copyright page.
